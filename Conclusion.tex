\chapter{Conclusion}
\section{Summary and conclusion}
We have derived analytically the velocity of a skyrmion under the effects of a spin-transfer torque from a spin polarized current, Rashba spin--orbit coupling, a constant electric field gradient and a phenomenological pinning force. It was found that the effects of the spin--orbit coupling could be viewed as a modification of the non-adiabatic spin-transfer torque. The motion of the skyrmion due to a constant electric field gradient could be described as a motion along the equipotential lines perpendicular to the gradient, and a damped motion proportional to the Gilbert damping in the opposite direction of the gradient. The electric field gradient is not able to generate high skyrmion velocities, but can be used to cancel out the perpendicular velocity component due to the topological Hall effect in the motion driven by a spin-polarized current. In the presence of pinning forces it was found that the critical current was smaller for smaller skyrmions when Rashba spin--orbit coupling was present, but the reduction was only significant for large values of the non-adiabatic spin-transfer torque. Without the presence of Rashba spin--orbit coupling the situation is reversed, and larger skyrmions have a smaller critical current. For the motion driven by electric field gradients larger skyrmions had a lower critical field gradient.

The analytical derivation was based on the micromagnetic model and the Thiele equation. The Thiele equation is a simplification of the Landau--Lifshitz--Gilbert equation, and assumes a rigid motion of the skyrmion. To check if the ansatz of a rigid motion holds true the LLG equation was also solved numerically for comparison to the results from the Thiele equation, as the LLG equation makes no assumptions regarding the shape of the skyrmion during its motion. Due to the complexity of the LLG equation, however, stability problems were encountered when utilizing a standard finite element method solver. After a certain time step the numerical simulations did not longer conserve the magnitude of the magnetization, which is explicitly conserved by the form of the LLG equation. To attempt to get any useful results before this instability rendered any result useless, very strong currents and electric field gradients were considered to be able to observe the dynamics of the skyrmion at that time-scale. The results observed agreed qualitatively with the analytical results derived, but some deformation of the skyrmion was also observed. This could potentially be due to the extreme values used in the numerical simulations. To be able to draw any real conclusions from the numerical data it is therefore recommended to run new simulations with a solver more specialized for the LLG equation, so that it is possible to observe the skyrmion dynamics over longer time steps with more realistic parameters for the current and electric field gradient.

In addition to the electric control of the skyrmion dynamics, we also studied the electrical control of spin torque oscillators in systems with Rashba spin--orbit coupling in the free magnetic layers. The strength of this spin--orbit coupling was tunable by a spin polarized electric current, allowing us to control the magnitude of the resulting effects in our system. The phase diagrams for two different geometries were calculated analytically and numerically for varying strengths of the spin--orbit coupling. It was discovered that by having Rashba spin--orbit coupling in our system we could increase the size of the resulting spin torque oscillator phase in our phase diagrams, and that it was possible to get an oscillating phase in both ferromagnetically and antiferromagnetically coupled compensated magnetic layers, the latter of which being a new discovery. Moreover, the oscillations in the spin torque oscillator phase were often found to have higher amplitudes in their oscillations when spin--orbit coupling was present, and the frequencies obtained often differed slightly from the system without spin--orbit coupling. This is of particular interest as it allows for a wider tunability of the frequency output by varying the strength of the current applied to the system.

\section{Prospects for further studies}
In this thesis the focus has been on electrical control of magnetization dynamics. We have shown how the direction of the skyrmion motion can be adjusted by a constant electric field gradient, and previously it has also been shown how skyrmions can be guided by generating unstable regions with electric fields \cite{Upadhyaya2015}. What can also be of interest to study is how the motion of skyrmions are affected by more complex spatial variations in the electric field. If it was possible to create an energy landscape that allowed us to generate a trajectory for individual skyrmions this would be highly attractive in skyrmion-based memory applications. 

The motion of skyrmions can also be manipulated by other means than electric currents and fields. In recent years it has been shown that skyrmion motion can be driven by magnons, both in a translational \cite{Iwasaki2014} and rotational \cite{Mochizuki2014} manner. The motion is induced by the scattering of the magnons in the skyrmion. The translational motion can be generated by a spin-wave source, whereas the rotational motion has been found to occur due to thermal magnons. How the skyrmion magnon-driven motion can be controlled in a manner viable for use in memory application is still a topic of interest. 

The magnon-driven magnetization dynamics has also been shown to be able to drive spin torque oscillators in ferromagnetic multilayers \cite{Bender2016}, but has yet to be studied for an antiferromagnetic coupling.