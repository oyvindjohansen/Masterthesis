\chapter{Introduction}
% Magnetization dynamics in memory and logic devices
Most of the technological devices we use in our daily lives today are based on electronics, where we utilize the charge of the electron to transfer and process information. In recent decades an alternative to electronics has emerged, and its name is spintronics. Spintronic devices utilize the spin of the electron instead of its charge. These devices have several advantages over electronic devices, such as a lower power consumption and that the materials needed to realize the devices are fairly common metals \cite{Okamoto2014}, where electronic devices need rarer semiconductor materials to function. Spintronic devices are widely used in non-volatile memory applications, such as magnetoresistive random access memory (MRAM) \cite{Akerman2005,Katine2008}, racetrack memories \cite{Parkin2008} and spin-transfer torque magnetic random access memory (STT-MRAM) \cite{Kent2015}. Most of these applications are based on the giant magnetoresistance effect (GMR), which makes the resistance of a spin-polarized electrical current passing through a magnetic multilayer system dependent on the orientation of the magnetic layers. When the magnetic moments of the different layers were anti-aligned the resistance of the current passing through the layers was much greater than the resistance of a current passing through magnetic layers with aligned magnetic moments. The discovery of this effect by Fert \cite{Fert1988} and Gr\"{u}nberg \cite{Grunberg1989} in 1988 was awarded by the Nobel prize in physics in 2007. This effect makes it possible to use magnetic multilayers for information storage in the form of bits, as we have a high-resistance and a low-resistance state, which can be assigned to the values zero and one. In addition, this information is not necessarily lost in the absence of an applied electric current or magnetic field, as the magnetic layers can keep their orientation in an equilibrium state. This makes the spintronic devices particularly interesting for non-volatile memory applications.

Being able store information in magnetic systems is not too interesting by itself, we also need to be able to read and write the information effectively. This is where magnetization dynamics comes into play, a topic that this thesis will mainly revolve around. While it is known that magnetic moments can be manipulated by magnetic fields, this method is often impractical for nano-scale devices due to the large fields required to manipulate the magnetic moments in a very localized region. It has been discovered, however, that the magnetization in a magnetic material can also be manipulated by electrical means. An experiment by Johnson and Silsbee in 1985 showed that the spin direction of the electrons in a current passing through a magnetic layer was aligned with the magnetic moment in the layer \cite{JohnsonSilsbee1985}. The magnetization in the film must therefore have exerted a torque on the spins of the conduction electrons. It is then reasonable to think that the spins of the conduction electrons will also exert a torque on the magnetization in the magnetic layer. This was shown theoretically by Berger \cite{Berger1996} and Sloncewski \cite{Slonczewski1996}, and later observed experimentally \cite{Tsoi1998}. This torque, known as the spin-transfer torque, could be used to switch the magnetization direction of one of the magnetic layers in a multilayer system \cite{Myers1999,Sun1999,Katine2000}. It is therefore a very useful mechanism in writing techniques in spintronic devices.

% The use of skyrmions vs domain walls (critical current, Joule heating, storage density)

% Applications of spin torque oscillators

% Interest in RSOC and its applications