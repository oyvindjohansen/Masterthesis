\chapter{Introduction}
% Magnetization dynamics in memory and logic devices
Most of the technological devices we use in our daily lives today are based on electronics, where we utilize the charge of the electron to transfer and process information. In recent decades an alternative to electronics has emerged, and its name is spintronics. Spintronic devices utilize the spin of the electron instead of its charge. These devices have several advantages over electronic devices, such as a lower power consumption and that the materials needed to realize the devices are fairly common metals \cite{Okamoto2014}, whereas electronic devices need rarer semiconductor materials to function. Spintronic devices are widely used in non-volatile memory applications, such as magnetoresistive random access memory (MRAM) \cite{Akerman2005,Katine2008}, racetrack memories \cite{Parkin2008} and spin-transfer torque magnetic random access memory (STT-MRAM) \cite{Kent2015}. Most of these applications are based on the giant magnetoresistance effect (GMR), which is that the resistance of a spin-polarized electrical current passing through a magnetic multilayer system is dependent on the orientation of the magnetic layers. When the magnetic moments of the different layers are anti-aligned the resistance of the current passing through the layers is much greater than the resistance of a current passing through magnetic layers with aligned magnetic moments. The discovery of this effect by Fert \cite{Fert1988} and Gr\"{u}nberg \cite{Grunberg1989} in 1988 was awarded by the Nobel prize in physics in 2007. This effect makes it possible to use magnetic multilayers for information storage in the form of bits, as we have a high-resistance and a low-resistance state, which can be assigned to the values zero and one. In addition, this information is not necessarily lost in the absence of an applied electric current or magnetic field, as the magnetic layers can keep their orientation in an equilibrium state. This makes the spintronic devices particularly interesting for non-volatile memory applications.

While the prospect of storing information in magnetic systems is interesting in its own right, we also need to be able to read and write the information effectively. This is where magnetization dynamics comes into play, a topic that this thesis will mainly revolve around. While it is known that magnetic moments can be manipulated by magnetic fields, this method is often impractical for nano-scale devices due to the large fields required to manipulate the magnetic moments in a very localized region. It has been discovered, however, that the magnetization in a magnetic material can also be manipulated by electrical means. An experiment by Johnson and Silsbee in 1985 showed that the spin direction of the electrons in a current passing through a magnetic layer was aligned with the magnetic moment in the layer \cite{JohnsonSilsbee1985}. The magnetization in the film must therefore have exerted a torque on the spins of the conduction electrons. It is then reasonable to think that the spins of the conduction electrons will also exert a torque on the magnetization in the magnetic layer. This was shown theoretically by Berger \cite{Berger1978,Berger1984,Berger1992,Berger1996} and Sloncewski \cite{Slonczewski1996}, and later observed experimentally \cite{Tsoi1998}. This torque, known as the spin-transfer torque, could be used to switch the magnetization direction of one of the magnetic layers in a multilayer system \cite{Myers1999,Sun1999,Katine2000}. It is therefore a very useful mechanism in writing techniques in spintronic devices. 

% The use of skyrmions vs domain walls (critical current, Joule heating, storage density)
Spin-transfer torques have also been shown to induce a translational motion of magnetic textures, such as domain wall motion \cite{Gan2000,Vernier2004}. This is very useful in information processing, as we can move the information encoded in magnetic domains without any motion of physical components. This is utilized in racetrack memories \cite{Parkin2008}. This motion of magnetic textures is not without problems, however. Ideally we would move these magnetic textures in a purely translational manner, but above some critical current strength the motion of magnetic domain walls has been shown to undergo a Walker breakdown \cite{SchryerWalker1974}. This Walker breakdown is an excitation of a different type of dynamics of the domain wall, in the form of rotational motion. In addition, at the current density needed to move domain walls Joule heating becomes a concern \cite{Yamaguchi2005}, causing energy loss to heat generation and thermal effects in the domain wall motion. Another magnetic texture known as the magnetic skyrmion has been of considerable interest lately, however. This is due to the fact that the critical current necessary to move them has been shown to be as low as $10^6$ A/m$^2$ \cite{Jonietz2010,Schulz2012,Yu2012}, whereas domain walls have a critical current in the range $10^{10}-10^{12}$ A/m$^2$ \cite{Grollier2003,Parkin2008}. This makes it possible to move them without considerable effects from Joule heating. In multiferroic materials it is even possible to create skyrmion based memory devices without any Joule heating effects at all \cite{Mochizuki2015}, as they can be controlled by electric fields via coupling to the electric dipole moment of the skyrmion. The dynamics of skyrmions has also been shown to be controllable by electric fields even in ferromagnetic materials \cite{Upadhyaya2015}.  Skyrmions are topologically stable objects, meaning  they are also suitable for non-volatile memory applications. They are also of particular interest due to their ability for an extremely dense storage of information \cite{Fert2013}, as a single skyrmion can span just a few nanometers. In addition, methods that would make it possible to nucleate skyrmions by injection of a spin-polarized current \cite{Sampaio2013}, spin-waves \cite{Liu2015} or by applying mechanical stress \cite{Nii2015} have been suggested. This ability to manipulate skyrmions makes them very attractive in memory applications. The downside of skyrmions with respect to magnetic domain walls is that the skyrmion velocities obtained so far are well below what is achievable with magnetic domain walls \cite{Fert2013,Yang2015}. 

% Applications of spin torque oscillators
A third use of the spin-transfer torque, in addition to magnetization switching and translation of magnetic textures, is the precession of magnetic moments it can induce in magnetic multilayer systems \cite{Tsoi2000,Kiselev2003,Rippard2004}. When the magnetic moments in a multilayer system precess with respect to one another, so the projection of one magnetic moment onto another varies in time, this will cause an oscillation in the resistance of a current passing through the system due to the giant magnetoresistance effect. A direct current passing through the multilayer system will then be transformed into an alternating current. Such a system with a self-sustained precession in the magnetic layers is known as a spin torque oscillator \cite{Silva2008,Kim2012STO}. These oscillators are able to generate a wide range of frequencies in the output signal, spanning the range of 100s of MHz to 100s of GHz \cite{Katine2008,Silva2008,Sun2008}. In addition, the frequency obtained can easily be tuned by the strength of the applied current. This type of self-sustained precession has been shown to occur in both ferromagnetically \cite{Zhou2013} and antiferromagnetically \cite{Klein2012} coupled magnetic layers, although Ref. \cite{Zhou2013} was unable to reproduce the results of Ref. \cite{Klein2012}. 

% Interest in RSOC and its applications
Both skyrmions and spin torque oscillators appear in systems with a broken inversion symmetry. Skyrmions appear in chiral magnets, while spin torque oscillators are based magnetic multilayer systems, which have a broken inversion symmetry at the interfaces of the different layers. When we have a system with broken inversion symmetry, Rashba spin--orbit coupling \cite{BychovRashba1984,Heide2006} can arise and a current passing perpendicularly to the direction of the asymmetry will induce spin--orbit torques. This type of spin--orbit coupling has proven to cause interesting effects in many different areas \cite{Manchon2015}, such as the appearance of an intrinsic spin Hall effect \cite{Sinova2004}. The spin--orbit coupling was also an important mechanism in the implementation of the spin field-effect transistor by Datta and Das \cite{DattaDas1990}. One of the advantages of Rashba spin--orbit coupling is its ability to be modified by the application of an electric field via gate voltages \cite{Schultz1996,Nitta1997}. This way the strength of the inversion asymmetry can be controlled to some extent. A discovery that is of particular interest to us is that Rashba spin--orbit coupling can also contribute to magnetization switching and self-sustained oscillations in spin torque oscillators \cite{Miron2011,Duan2014}.

% General structure of thesis
In this thesis we will study the magnetization dynamics of a single skyrmion (which is driven by spin-transfer torques and inhomogenous electric fields) and spin torque oscillators, both in the presence of Rashba spin--orbit coupling. We first introduce the micromagnetic model and the Landau--Lifshitz--Gilbert equation as a foundation for our calculations. Before deriving the equations of motion for the skyrmion from the Thiele equation, we briefly discuss the structure, symmetries and dynamical properties of the skyrmion. The analytical solution of the Thiele equation is then compared to a full numerical solution of the Landau--Lifshitz--Gilbert equation. Lastly, we combine analytical and numerical calculations to find the spin torque oscillator state and its frequency spectrum based on the Landau--Lifshitz--Gilbert--Slonczewski equation. 